\documentclass[12pt,twoside]{article}

\usepackage[utf8]{inputenc}
\usepackage{mathtools}
\usepackage{ulem}
\usepackage{mathcomp}
\usepackage[italian]{babel}
\usepackage[margin=1in]{geometry} 
\usepackage{scrextend}
\usepackage{relsize}
\usepackage{amsmath,amsthm,amssymb,amsfonts}
\usepackage{sectsty}


\newcommand{\N}{\mathbb{N}}
\newcommand{\Z}{\mathbb{Z}}
\newcommand{\R}{\mathbb{R}}
\newcommand{\sesolose}{\Leftrightarrow}
\newcommand{\implica}{\Longrightarrow}
\newcommand{\nin}{\forall n\in\N}
\newcommand{\eps}{\varepsilon}
\newcommand{\pr}{\prime}
\newcommand{\pq}{\text{ per qualche }}
\renewcommand\qedsymbol{$\blacksquare$}
\newcommand{\stkout}[1]{\ifmmode\text{\sout{\ensuremath{#1}}}\else\sout{#1}\fi}
\newcommand{\znz}{\displaystyle({\Z}/{n\Z})^*}
\newcommand{\znzbig}{\displaystyle\bigg(\frac{\Z}{n\Z}\bigg)^*}
\newcommand{\Eps}{${\Large$\epsilon$}$}
\newcommand{\grafo}{(V, \: \Eps)}
\newcommand{\til}{\raise.17ex\hbox{$\scriptstyle\mathtt{\sim}$}}
\newcommand{\baseinduz}[1]{\newline\newline {\boldmath$n = #1$}}
\newcommand{\induzuno}[1]{\newline\newline {\boldmath$n \ge #1, n \implica n+1$}}
\newcommand{\induzdue}[1]{\\\\{\boldmath$n \ge #1, \forall k < n \implica n$}}
\newcommand{\baseinduzalbero}[1]{\\\\{\boldmath$ |V(T)| = #1$}}
\newcommand{\induzalbero}[1]{\\\\{\boldmath$|V(T)| \ge #1, |V(T)| -1 \implica |V(T)|$}}

\DeclareMathAlphabet{\pazocal}{OMS}{zplm}{m}{n}
 
\newtheorem{theorem}{Teorema}

\makeatletter
\def\l@section{\@dottedtocline{1}{0em}{3em}}
\makeatother

\renewcommand{\thesection}{\Roman{section}} 
\sectionfont{\large}

\hbadness=0
 
\begin{document}
 
\title{Teoremi richiesti all'Esame di\\Fondamenti matematici per l'informatica}
\author{Matteo Franzil}
\maketitle

\tableofcontents

\clearpage 

\section{Buon ordinamento dei numeri naturali e seconda forma del principio di induzione}
\begin{theorem}[Buon ordinamento dei numeri naturali]
($\mathbb{N}, \le$) è ben ordinato.
\end{theorem}
 
\begin{proof}
Supponiamo esista $A\subset\N$ dove $\nexists \min A$.
 Sia $B \coloneqq\N \backslash A$. Dimostriamo che $B = \N$ e A$ = \emptyset$.
 Procediamo per induzione di prima forma. Sia $\{0, 1, \ldots ,n\} \subset B \ \forall n\in\N$, ovvero $P(n) = (\{0, 1, \ldots ,n\} \subset B)$ è vera $\nin$.
\baseinduz{0}
\\$\{0\}\subset B \sesolose 0 \in B \sesolose 0 \not\in A$.
\\Se supponessimo per assurdo che $0 \in A$, allora avremmo che $0 = \min A$. Quindi $0 \not\in A$.
\induzuno{1}
\\Assumiamo che $\{0, 1, \ldots, n\} \subset B$ per qualche $n$.
\\Proviamo che $\{0, 1, \ldots, n, n+1\} \subset B$.
\\$n+1\subset A$? No, perché altrimenti avremmo che $n+1 = \min A$.
\\Allora
$$n+1 \in B \implica B = \N, \ A = \emptyset$$
\end{proof}

\begin{theorem}[Seconda forma del principio di induzione]
Sia una famiglia di proposizioni $\{P(n)\}_{n\in\N}$ indicizzata su $n\in\N$. Supponiamo che 
\begin{enumerate}
\item $P(0)$ è vera
\item $\forall n > 0$, $(P(k) $ è vera $ \forall k < n) \implica P(n)$ è vera.
\end{enumerate}
Allora $P(n)$ è vera $\nin$.
\end{theorem}

\begin{proof}
Sia $A \coloneqq \{n\in\N| P(n)$ è falsa$\}$, dimostriamo che $A = \emptyset$.
\\Supponiamo che:
$$A \ne \emptyset \implica \exists n \in \N : n = \min A. \text{ Per la (1), essendo P(0) vera, } n \ne 0$$
Inoltre, se $k < n$, $k \not\in A$ in quanto abbiamo che $n = \min A$, ma allora dalla $(2)$ segue che $P(n)$ è vera e che quindi $n \not\in A$, che è in contraddizione con quanto asserito all'inizio della dimostrazione.
\end{proof}

\section{Esistenza e unicità della divisione euclidea}
\begin{theorem}[Esistenza e unicità della divisione euclidea]
Siano $n, m \in \Z$ con $m \ne 0$.
\\$\implica \exists ! $q, r $\in \Z :$
\begin{itemize}
\item$n = qm + r$
\item$0 \le r < |m|$
\end{itemize}
\end{theorem}

\renewcommand\qedsymbol{$\square$}
\begin{proof}[Esistenza.]
Procediamo per induzione di seconda forma su $n$.
\baseinduz{0}
\\Poniamo $q, r = 0$.
\induzdue{1}
\\Supponiamo $n > 0$ e l'asserto vero $\forall k < n$. Dimostriamo che l'asserto vale $\nin$.
\begin{itemize}
\item Consideriamo innanzitutto il caso $n \ge 0$. Se $n < m$, poniamo $q \coloneqq 0$ e $ r \coloneqq n$.
\item Altrimenti, avremo che $n \ge m$. Sia $k \coloneqq n - m$.\\Applicando la divisione euclidea, otteniamo che:
\begin{align*}
\exists q, r \in \N : k &= mq + n, \ \ 0 \le k <n, \\
\sesolose n = k + m =&\ (qm + r) + m = (q+1) m + r.
\end{align*}
\item Analizziamo ora il caso opposto, ovvero quando $n < 0$. Se $m > 0$, applicando la procedura di divisione euclidea a $-n > 0, m >0$, vale:
\begin{align*}
\exists q, r \in \N:\ -n =&  \ qm + r, \ \ 0 \le r < |m| \\
\sesolose n =& - qm - r.
\end{align*}
Se $ r = 0$ abbiamo \sout{vinto} finito, altrimenti continiuamo per ottenere un resto $> 0$. Aggiungendo e togliendo $m$:
\begin{align*}
n &= (-q)-r-m+m \\
&= (-q-1)m+(m-r)
\end{align*}
dove $m-r$ è strettamente positivo per definizione.
\item Sia infine $m < 0$, ovvero $-m > 0$.
\begin{align*}
\implica \exists q, r \in \Z:\ n &= (-m)q + r, \ \ 0 \le r < |m| \\
 \sesolose n &= (-q)m + r 
\end{align*}
\end{itemize}
\end{proof}
\renewcommand\qedsymbol{$\blacksquare$}
\begin{proof}[Unicità.]
Supponiamo $\exists n, m \in \Z$, $m \ne 0$; $q, q^{\pr}, r, r^{\pr} \in \N$:
\begin{align*}
n = qm + r,& \quad 0 \le r < |m| \\
n = q^{\pr}m + r^{\pr},& \quad 0 \le r^{\pr} < |m|
\end{align*}
Proviamo che $ q = q^{\pr}, r = r^{\pr}$. Possiamo supporre che $r^{\pr} > r$. Allora vale:
\\$ qm - q^{\pr}m = r^{\pr} - r \sesolose m(q-q^{\pr}) = r^{\pr}-r$. Effettuando l'operazione di modulo otteniamo:
$$ |m(q-q^{\pr})| = |r^{\pr} - r| = r^{\pr} - r < |m|$$
Affinché la disuguaglianza sia rispettata deve essere $ 0 \le |q-q^{\pr}| < 1 $.
\\ Essendo $q, q^{\pr} \in \N$, concludiamo che $q^{\pr} - q = 0 \implica q^{\pr} = q$.
\\ Dall'equazione originale ricaviamo infine che: $mq + r = mq^{\pr} + r^{\pr} \implica r^{\pr} = r$.
\end{proof}

\section{Unicità della rappresentazione di un numero in base arbitraria}
\begin{theorem}[Unicità della rappresentazione di un numero in base $b \ge 2$ arbitraria]
Sia $b \in \N, b \ge 2 \implica \nin$, $\exists!$ rappresentazione di $n$ in base $b$, ovvero una successione $\{\eps_{i}\}$ con le seguenti proprietà:
\begin{enumerate}
\item $\{\eps_i\}_{i\in\N}$ è definitivamente nulla dopo qualche $i_{0}\in\N$, ovvero $\forall j \ge i_{0}, \eps_{j} = 0$.
\item $\eps_i \in I_b = \{0, 1,\ldots ,b-1\}$ $\forall i \in \N$ (ovvero $0 \le \eps_i < b$)
\item $\displaystyle\sum_{i\in\N} \eps_{i}b^i = n$
\end{enumerate}
Inoltre, se esiste un altra successione $\{\eps_i'\}_{i\in\N}$ allora $\eps_i = \eps_{i}^{\pr}$ $\forall i \in \N$.
\end{theorem}

\renewcommand\qedsymbol{$\square$}
\begin{proof}[Esistenza.]
Procediamo per induzione di seconda forma su n.
\baseinduz{0}
\\Vale:
$$ n = \sum_{i\in\N} \eps_i b_i = \sum_{i\in\N} 0 b_i = 0_b \ \forall i \in\N$$
\induzdue{1}
\\Supponiamo $n > 0$ e l'asserto vero $\forall k < n$.
\\Eseguiamo la divisione euclidea di $n$ con $b$:
$$n = qb + r, \qquad0 \le r < |b|$$
Per ipotesi sappiamo che $b \ge 2$, quindi vale $ 0 < q < qb \le qb + r = n$.
\\Per ipotesi induttiva allora esiste una successione $\{\delta_i\}$ che possiede le proprietà $(1)$, $(2)$, $(3)$; inoltre vale:
\begin{align*}
n &= \Big(\sum \delta_i b^i\Big) b + r\\
n &= \Big(\sum \delta_i b^{i+1}\Big) + r
\end{align*}
Sia ora $r = \eps_0$; effettuando un cambio di indice, otteniamo:
$$ n = \eps_0 + \sum_{j \ge 1} \delta_{j-1}b^j = \eps_0 + \delta_0 b^1 + \delta_1 b^2 + \ldots = \sum_{i\in\N} \eps_i b^i $$
\end{proof}

\renewcommand\qedsymbol{$\blacksquare$}
\begin{proof}[Unicità.]
Procediamo per induzione di seconda forma.
\baseinduz{0}
\\Se $n = 0$ allora tutti gli addendi della sommatoria saranno nulli $\implica \eps_i = 0$ $\forall i \in \N$.
\induzdue{1}
\\ Sia $n > 0$. Assumiamo l'asserto sia vero $\forall k < n$ e dimostriamo che $P(n)$ è verificata $\nin$.
Assumiamo esistano $\{\eps_i\}_{i\in\N}, \{\eps_i'\}_{i' \in\N}$ con le proprietà $(1)$, $(2)$, $(3)$.\\Proviamo che $\eps_i = \eps_{i}^{\pr}$ $\forall i \in \N$.
\\Osserviamo che:
$$n =\sum_{i\in\N} \eps_i b^i = \eps_0 + b\Bigg( \sum_{i \ge 1} \eps_i b^{i-1}\Bigg)$$
$$n =\sum_{i\in\N} \eps_{i}^{\pr} b^i = \eps_{0}^{\pr} + b\Bigg( \sum_{i \ge 1} \eps_{i}^{\pr} b^{i-1}\Bigg)$$
dove $\eps_{0}^{\pr},\eps_{0}$ sono i resti delle divisioni di n per b. Ma per l'unicità dellla divisione euclidea vale $\eps_{0}^{\pr} = \eps_{0}$. Stesso discorso per i quozienti, che inoltre risultano per definizione $\le n$. Segue, cambiando gli indici della sommatoria:
\\$$q = \sum_{j\in\N} \eps_{j+1}^{\pr} b^j = \sum_{j\in\N} \eps_{j+1} b^j < n $$
Come prima si ha $q < n$ e per ipotesi di induzione si ha che $\eps_i = \eps_{i}^{\pr}$ $\forall i \ge 1$
\end{proof}

\section{Esistenza e unicità del Massimo Comune Divisore e del minimo comune multiplo}
\begin{theorem}[Esistenza e unicità del Massimo Comune Divisore]
Siano $n, m \in \N$ con $n, m$ non entrambi nulli. Diremo che un $d \in \N, d \ge 1$ è Massimo Comune Divisore (M.C.D.) di $n, m$ se:
\begin{enumerate}
\item$d | m \land d | n $
\item$c | m \land c | n \implica c | d$ per qualche $c \in \N$.
\end{enumerate}
Inoltre, $\exists x, y \in \Z : d = xn + ym$, ovvero $d$ è esprimibile come combinazione lineare con $x, y$.  Se $\exists$ MCD tra $n, m$, è unico e lo indicheremo con $(n, m)$.
\end{theorem}

\renewcommand\qedsymbol{$\square$}
\begin{proof}[Unicità.]
Poniamo $\exists$ $d_1, d_2$ entrambi MCD di $n, m$. Applicando la proprietà $(1)$ di $d_1$ e la $(2)$ di $d_2$ otteniamo:
\begin{align*}
(1)\qquad& d_1 | m \land d_1 | n \\
(2)\qquad& \text{dato }c = d_1, \ \ d_1 | m \land d_1 | n \implica d_1 | d_2
\end{align*}
Applicando l'inverso otteniamo che $d_2 | d_1\ \land d_1 | d_2 \implica d_1 = \pm d_2$; essendo $d_1, d_2 \in \N$, otteniamo che
$ d_1 = d_2$.
\end{proof}

\renewcommand\qedsymbol{$\blacksquare$}
\begin{proof}[Esistenza.]
Sia $S \coloneqq \{ s \in \Z | s > 0, s = xn + ym \text{ per qualche } x, y \in \Z\}$.
\\Osserviamo che $S \ne \emptyset$, in quanto $nn + mm > 0, nn + mm \in S$.
\\Sia $d \coloneqq \min S$. Vale:
$$ d | m \land d | n ,\ \ \exists c \in \Z : (c | m \land c | n \implica c | d)$$
Essendo $d \in S, d = xn+ym \pq x, y \in Z$.
\\Dalla proprietà 2 si cha che $c | xn + ym$. Dimostriamo che $d | n$. Eseguendo la divisione euclidea tra $n, d$ otteniamo:
$$n = dq + r, \text{ } 0 \le r < |d|$$
Proviamo per assurdo che $r = 0$. Se fosse $r > 0$, avremmo che $r \in S$ (quindi risulterebbe che $d \ne \min S$, in quanto $d > r$). Vale:
\begin{align*} 
r = n - qd &= n - q(xn+ym) \\
&= n - qnx - qmy \\
&= n\underbrace{(1-qx)}_{x^{\pr}} + m\underbrace{(-qy)}_{y^{\pr}}
\end{align*}
Allora $r \in S$, ma ciò è assurdo. Quindi $r \ne S$.
\end{proof}

\begin{theorem}[Esistenza e unicità del minimo comune multiplo]
Siano $n, m \in \N$. Diremo che un $M \in \N$ è minimo comune multiplo di $n, m$ se:
\begin{enumerate}
\item$n | M \land m | M $
\item$n | c \land m | c \implica M | c$ per qualche $c \in \N$
\end{enumerate}
Se esiste, è unico lo indicheremo come $[m, n]$. Inoltre, se $n, m$ non sono entrambi nulli, vale:
$$[n, m] = \frac{nm}{(n, m)}$$
Se $n, m = 0$, allora $ [n, m] = 0$.
\end{theorem}

\renewcommand\qedsymbol{$\square$}
\begin{proof}[Unicità.]
Supponiamo esistano $M_1, M_2 \in \N : M_1, M_2$ sono entrambi mcm di $n, m$.
Applicando la proprietà $(1)$ di $M_1$ e la $(2)$ di $M_2$ otteniamo:
\begin{align*}
(1)\qquad& n | M_1 \land m | M_1 \\
(2)\qquad&\text{con } c = M_1 \text{, } \ \ n | M_1 \land m | M_1 \implica M_2 | M_1
\end{align*}
Invertendo le proprietà si ha che $M_1 | M_2 \land M_2 | M_1 \implica M_2 = \pm M_1$.
\\Essendo $M_1, M_2 \in \N$, $M_2 = M_1$.
\end{proof}


\renewcommand\qedsymbol{$\blacksquare$}
\begin{proof}[Esistenza.]
Supponiamo $n, m$ non entrambi nulli. Osservo che
\begin{align*}
(n, m)|n \quad\sesolose&\quad n = n^{\pr}(n,m) \pq n^{\pr}\in\Z \\
(n, m)|m \quad\sesolose&\quad m = m^{\pr}(n,m) \pq m^{\pr}\in\Z
\end{align*}
Definisco $\displaystyle M \coloneqq \frac{nm}{(n, m)}$. Sostituendo si ha che
\begin{align*}
\frac{nm}{(n, m)}\ = \ \frac{n^{\pr}m^{\pr}(n, m)(n, m)}{(n, m)} \quad=&\quad n^{\pr}m^{\pr}(n,m) \\
\quad=&\quad (n^{\pr}(n, m))m^{\pr} = nm^{\pr} \\
\quad=&\quad (m^{\pr}(n,m))n^{\pr} = mn^{\pr}
\end{align*}
Allora $n | M, m | M$. Sia ora $c \in \Z$. Verifichiamo la $(2)$, ovvero che $n | c \land m | c \stackrel{?}{\implica} M | c$.
\\ Vale:
\begin{align*}
(n, m) | n,\ n|c &\implica (n,m)|c \\
(n, m) | m,\ m|c &\implica (n,m)|c 
\end{align*}
Allora $c = c^{\pr}(n, m) \pq c^{\pr} \in \Z$.
\\Sappiamo infine che $(n^{\pr}, m^{\pr}) = 1$; per definizione abbiamo che $n^{\pr} | c^{\pr} \land m^{\pr} | c^{\pr} \implica n^{\pr}m^{\pr} | c^{\pr}$.
\\Moltiplicando e sinistra a destra si ottiene
$$ \underbrace{n^{\pr}m^{\pr}(n,m)}_M | \underbrace{c^{\pr}(n,m)}_c$$
\end{proof}

\section{Teorema fondamentale dell'aritmetica}
\begin{theorem}[Teorema fondamentale dell'aritmetica]
Ogni $n \in \N, n \ge 2$ si può scrivere come prodotto finito di numeri primi:
$$ n = p_1 p_2 p_3 \cdots p_k\qquad p_1, p_2, \cdots, p_k \in \N \ \text{primi eventualmente ripetuti}$$
Tale scrittura è unica a meno di permutazioni. Se
$$ n = q_1 q_2 q_3 \cdots q_h\qquad q_1, q_2, \cdots, q_h \in \N \ \text{primi eventualmente ripetuti}$$
Allora $k = h$ ed $\exists \varphi : \{1, 2,\ldots ,k\} \mapsto \{1, 2, \ldots , h\}$, una bigezione (ovvero una permutazione su \{1, 2,\ldots ,k\}) tale che:
$$ p_i = q_{\varphi(i)}\quad \forall i \in \{1, 2,\ldots ,k\} $$
\end{theorem}

\renewcommand\qedsymbol{$\square$}
\begin{proof}[Esistenza.]
Procediamo per induzione di seconda forma.
\baseinduz{2}
\\Abbiamo che $ 2 = 2 $.
\induzdue{2}
\\Se n è primo \sout{si va al mare} abbiamo finito.
\\Altrimenti possiamo ipotizzare $n = d_1, d_2 :\enspace1 < d_1 < n,\enspace1 < d_2 < n$, dove
\begin{align*}
d_1 &= p_1 p_2 p_3 \cdots p_k \\
 d_2 &= p^{\pr}_1 p^{\pr}_2 p^{\pr}_3 \cdots p^{\pr}_k
\end{align*}
 per ipotesi di induzione. Allora n è fattorizzabile perché prodotto di numeri primi positivi.
\end{proof}

\renewcommand\qedsymbol{$\blacksquare$}
\begin{proof}[Unicità.]
Supponiamo che esistano due distinte fattorizzazioni:
\begin{align*}
n &= p_1 p_2 p_3 \cdots p_k \\
n &= q_1 q_2 q_3 \cdots q_h
\end{align*}
con $h \ge k$. Procediamo per induzione di prima forma.
\\\\{\boldmath$k = 1$}
\\Vale $p_1 = n = q_1 q_2 \cdots q_h$ con $h \ge 1$. Dimostriamo che $h = 1$. Ipotizziamo per assurdo che $ h \ge 2$; avremmo che $ n = q_1 q_2 \cdots q_h$. Sappiamo che essendo $p_1$ primo, deve necessariamente essere $ q_j = 1 \lor q_j = p_1$; tuttavia per ipotesi abbiamo che $q_j > 1 \implica q_j = p_1$.
\\Allora si ha che
$$ p_1 = n = q_1 q_2 \cdots q_h \ge q_1 q_2 = p^2_1 > p_1 = n $$
che è un assurdo ($n \ngtr n$). Allora $h = 1$.
\\\\{\boldmath$k \ge 2, k \implica k+1$}
\\Con $ k > 1 $, assumiamo l'asserto vero per k ($n = p_1 p_2 \cdots p_k = q_1 q_2 \cdots q_h$ con $h = k$, $p_i = q_i \quad\forall i \in \N$ a meno di permutazioni) e dimostriamolo per $k + 1 = h$. Supponiamo quindi che $p_1 p_2 \cdots p_k p_{k+1} = q_1 q_2 \cdots q_h$ con $h > k + 1$. Abbiamo che $p_{k+1} | n \implica p_{k+1} | q_1 q_2 \cdots q_h$, allora $p_{k+1} | q_h$ per ipotesi; essendo $p_{k+1}, q_h$ primi positivi, vale $p_{k+1} = q_h$. Ma allora 
$$p_1 p_2 \cdots p_k = q_1 q_2 \cdots q_{h-1} $$
dove entrambi i membri sono stati divisi per $p_{k+1}$. Ma allora per ipotesi d'induzione le due fattorizzazioni hanno lo stesso numero d'elementi, ovvero 
$$k = h - 1, e p_1 = q_1, p_2 = q_2, \cdots p_{k+1} = q_h$$
\end{proof}

\section{Teorema cinese del resto}
\begin{theorem}[Teorema cinese del resto]
Siano $n, m \in \N; a, b \in \Z$. Consideriamo il seguente sistema di congruenze:
\[
S = \left\{
\setlength\arraycolsep{2pt}
\begin{array}{rclc} x& \in &\Z & \\ x & \equiv &a\quad (mod\ n) &\qquad (1) \\ x & \equiv& b\quad (mod \ m)& \qquad (2) 
\end{array}\right
.
\]
Definiamo $Sol(S) \coloneqq \{ x \in \Z \ |\ (1), (2)$ sono verificate$\}$.
\\ $Sol(S) \ne \emptyset \ \sesolose$ S è compatibile $ \sesolose \ (n,m) | (a-b)$.
\\ Se S è compatibile, data $c \in \Z$ soluzione particolare di $S$, vale:
$$Sol(S) = [c]_{[n,m]} = \{ c + k[n,m] \in \Z \ |\ k \in \Z\}$$
\end{theorem}

\renewcommand\qedsymbol{$\square$}
\begin{proof}[Dimostrazione (compatibilità)]\ \\
$(\implica)$.
Supponiamo $Sol(S) \ne \emptyset$. Sia $c \in Sol(S)$. Dimostriamo che valgono $(1), (2)$, ovvero $c \equiv a\ (mod \ n) \land c \equiv b\ (mod \ m)$. Riscriviamo il sistema di congruenze come:
\begin{align*}c &= a + kn \pq k \in \Z \\
c &= b + hm \pq h \in \Z
\end{align*}
Sottraendo membro a membro otteniamo:
$$ (a-b) + (kn - hm) = 0 \sesolose hm-kn = a-b $$
Sappiamo che $(n, m)|n \land (n,m)|m \implica (n,m)|(an+bm)$ dove $an + bm$ è una combinazione lineare di $n, m$ con qualche $a, b \in \Z$. Allora $(n, m) | (hm - km) = (a - b)$.
\end{proof}

\begin{proof}[$(\Longleftarrow)$]
Ora supponiamo $(n, m) | (a - b)$ sia vera, ovvero $a -b = k(n,m) \pq k \in \Z$. Applichiamo l'Algoritmo di Euclide a ritroso, ottenendo $(n, m) = xn + ym \pq x, y \in \Z$. Segue che:
$$ a - b = kxn + kym \quad\sesolose\quad \underbrace{a + (-kx)n}_c  =  \underbrace{b +(ky)m}_c $$
\end{proof}

\begin{proof}[Dimostrazione (insieme delle soluzioni)]\ \\
Supponiamo infine $Sol(S) \ne \emptyset$, ovvero che il sistema di congruenze è verificato. Sia $c \in Sol(S)$. Dimostriamo che $Sol(S) = [c]_{[n,m]}$ verificando che uno contiene l'altro e viceversa.
\\\\$(\subset)$.
Sia $ c^{\pr} \in [c]_{[n,m]}$, allora $c^{\pr} = c + k[n,m] \pq k \in \Z$. Riscrivo il sistema come
\[
S = \left\{
\setlength\arraycolsep{2pt}
\begin{array}{rcl} [c]_n & = & [a]_n \\ \relax
 [c]_m & = & [b]_m
\end{array}\right
.
\]
Vale:
\begin{align*}
[c^{\pr}]_n &= [c + k[n,m]]_n \\ 
&= [c]_n + [k]_n [[n,m]]_n \\
&= [a]_n + \stkout{[k]_n [0]_n} \ \ \leftarrow\scriptstyle{ \text{$[n,m]$ multiplo di n}}
\end{align*}
Con un procedimento analogo si ottiene $[c^{\pr}]_m = [b]_m$.
\end{proof}
\renewcommand\qedsymbol{$\blacksquare$}
\begin{proof}[$(\supset)$]
Sia $c \in Sol(S)$. Vale:
\begin{align*}
c &= a+hn = b + km \\
c^{\pr} &= a+h^{\pr}n = b + k^{\pr}m
\end{align*}
per qualche $h, h^{\pr}, k, k^{\pr} \in \Z$. Sottraiamo membro a membro:
\begin{align*}
c^{\pr} - c = (h^{\pr} - h)n &= (k^{\pr} - k)m \\
\implica n | (c^{\pr} -c),\ m|(c^{\pr} - c) &\implica [n, m]|(c^{\pr} - c) \\
&\sesolose c^{\pr} \equiv c \ (mod \ [n, m]) \\
&\sesolose c^{\pr} \in [c]_{[n,m]}
\end{align*}
\end{proof}

\section{Teorema di Fermat-Eulero e crittografia RSA}
\textbf{Definizione} (Formula di Eulero)\textbf{.} \textit{Sia $n \in \N, n \ge 2: n = p^{m_1}_1 p^{m_2}_2 \cdots p^{m_k}_k$ con $p_1 \cdots p_k$ primi a due a due distinti. Vale:
\begin{align*}
\phi(n) &= \phi(p^{m_1}_1)\phi(p^{m_2}_2) \cdots \phi(p^{m_k}_k) \\
&= (p^{m_1}_1 - p^{m_1 - 1}_1) (p^{m_2}_2 - p^{m_2 - 1}_2) \cdots (p^{m_k}_k - p^{m_k - 1}_k)
\end{align*}
}

\renewcommand\qedsymbol{$\square$}
\noindent\textbf{Lemma.} \textit{Siano $\alpha, \beta \in \znz$. Allora:
\begin{enumerate}
\item $\quad \forall \alpha, \beta \in \znz,\qquad(\alpha\beta)^{-1} = \beta^{-1} \alpha^{-1}$
\item $\quad \forall \alpha^{-1} \in \znz,\qquad (\alpha^{-1})^{-1} = \alpha $
\end{enumerate}
}\begin{proof} Vale:
\begin{enumerate}
\item$ (\alpha\beta)(\beta^{-1}\alpha^{-1}) = \alpha(\beta\beta^{-1})\alpha^{-1} = \alpha[1]_n\alpha^{-1} = \alpha\alpha^{-1} = [1]_n$
\item$ (\alpha)(\alpha^{-1}) = [1]_n$
\end{enumerate}
\end{proof}


\begin{theorem}[Teorema di Fermat-Eulero]
Sia $n > 0$. $\forall [a]_n \in \znz$, vale: 
\begin{align*}
[a]_{n}^{\phi(n)} = [1]_n
\end{align*}
Equivalentemente:
\begin{align*}
a^{\phi(n)} \equiv 1 \ (mod \ n ), \qquad \forall a \in \Z, \text{ con } (a, n) = 1
\end{align*}
\end{theorem}
\renewcommand\qedsymbol{$\blacksquare$}
\begin{proof}
Definiamo:
\begin{align*}
L_{\alpha}:\quad \znzbig &\longrightarrow \znz \\
\beta \quad&\longmapsto \quad\alpha\beta
\end{align*}
$L_{\alpha}$ è ben definita per il lemma precedente.
Proviamo che $ L_{\alpha}$ è una bigezione. Mostriamo che è iniettiva (la surgettività è dimostrata perché gli insiemi di partenza e arrivo coincidono, conseguenza del Lemma dei Cassetti). Supponiamo $\exists \beta_1, \beta_2 \in \znz$:
\begin{align*}
\alpha\beta_1 = L_{\alpha}(\beta_1) &= L_{\alpha}(\beta_2) = \alpha\beta_2\\
\implica \beta_1 = (\alpha^{-1}\alpha)\beta_1 = (\alpha^{-1})&(\alpha\beta_1) = (\alpha^{-1})(\alpha\beta_2) = \beta_2
\end{align*}
Sia ora $\{\beta_1, \beta_2, \ldots , \beta_k\} = \znz$ con $k = \phi(n)$. Applicando $ L_{\alpha}$ si ottiene
$$\{\alpha\beta_1, \alpha\beta_2, \ldots , \alpha\beta_k\} = \alpha^k(\beta_1, \beta_2, \ldots , \beta_k)$$
Allora $ L_{\alpha}$ non è altro che una permutazione, per cui possiamo scrivere:
$$ (\beta_1, \beta_2, \ldots , \beta_k) = \alpha^k(\beta_1, \beta_2, \ldots , \beta_k)$$
Moltiplicando a destra e a sinistra per $\{\beta_k^{-1}, \beta_{k-1}^{-1}, \ldots , \beta_1^{-1}\}$ si ottiene:
$$ \alpha^k = 1$$
\end{proof}

\noindent
\textbf{Definizione.}
\textit{Siano $n>0, m>0$. Definiamo:
\begin{align*}
P_{m}:\quad \znz &\longrightarrow \znz \\
\alpha \quad&\longmapsto \quad\alpha^{m}
\end{align*}
ovvero $P_m(\alpha) \coloneqq \alpha^m \quad\forall \alpha \in \znz$. $P_m$ è ben definita grazie al Lemma precedente.}

\begin{theorem}[Teorema fondamentale della crittografia RSA]
Sia $c > 0: (c, \phi(n)) <= 1$ con n fissato; $d > 0: d \in [c]_{\phi(n)}^{-1}$. \\
Allora la funzione $P_c$ (analoga a $P_m$ nella Definizione precedente) è invertibile e la sua inversa è $P_{c}^{-1} = P_d$.\\
\end{theorem}


\begin{proof}
Sia $\alpha \in \znz$. Osserviamo che 
\begin{align*}
[d]_{\phi(n)}[c]_{\phi(n)} &= [dc]_{\phi(n)} = [1]_{\phi(n)} \\
\sesolose dc &\equiv 1 \ (mod \ \phi(n)) \\
\sesolose dc &= 1 + k\phi(n)\quad \pq k \in \Z
\end{align*}
Applicando contemporaneamente $P_c$ e $P_d$ otteniamo che 
$$ P_d(P_c(\alpha)) = (\alpha^c)^d = \alpha^{cd} = \alpha^{1+k\phi(n)} = \alpha(\alpha^{\phi(n)})^{k} $$
Per il Teorema di Fermat-Eulero ciò è equivalente a $\alpha \cdot 1^k = \alpha$. Allo stesso modo dimostro che $P_c(P_d(\beta)) = \beta$.
\end{proof}

\section{Teoremi sulla congiungibilità nei grafi}
\begin{theorem}[Teorema di equivalenza tra la congiungibilità con cammini e congiungibilità con passeggiate]
Siano $G = \grafo$; $v, w \in V(G)$. Allora $v, w'$ sono congiungibili tramite cammini se e solo se sono congiungibili tramite passeggiate.
\end{theorem}

\begin{proof}\ \\
$(\implica)$. Banale. Il cammino è una passeggiata per definizione.\\
$(\Longleftarrow)$. Supponiamo esista una passeggiata P che congiunge $v, w$. Sia $\mathcal{P}$ l'insieme di tutte le passeggiate che congiungono $v, w$. Osserviamo che $\mathcal{P} \ne \emptyset \ \ (P \in \mathcal{P})$.\\
Sia $A \coloneqq \{ \underbrace{\mathcal{L}(\bar{P})}_{\text{lati di $\bar{P}$}} \in \N | \bar{P} \in \mathcal{P})$. Abbiamo che $A \ne \emptyset$, infatti $\mathcal{L}(P) \in A$.\\
Grazie al teorema del buon ordinamento $(\N, \le)$, vale:
$$ \exists \min A = m \implica \exists P_0 \in \mathcal{P} : \mathcal{L}(P_0) = m \le \mathcal{L}(\bar{P}),\ \ \forall \bar{P} \in \mathcal{P}$$
ovvero esiste $\min A$, quindi esiste una passeggiata $\mathcal{P}$ con il minimo numero di lati.\\
Proviamo ora che $P_0$ è un cammino in $G$. Vale:
$$ P_0 = (v_0, v_1\ldots, v_n)\qquad v = v_0,\ w = v_n$$
Poniamo per assurdo che $P_0$ non sia un cammino, ovvero $\exists i, j \in \{0, 1,\ldots, n\} : \ i < j, v_i = v_j$.\\
Definiamo $P_1 \coloneqq (v_0, v_1, \ldots , v_{i-1}, v_i, v_j, v_{j+1},\ldots, vn) \in \mathcal{P}$ (ovvero $P_0$ alla quale sono stati tolti tutti i vertici tra $i$ e $j$). Vale:
$$\mathcal{L}(P_1) = \mathcal{L}(P_0) - (j - i) = m - (j - i) < m$$
Ma ciò è assurdo in quanto $P_0$ è già per definizione un cammino con il minimo numero di lati.
\end{proof}

\begin{theorem}[La relazione di congiungibilità è una relazione di equivalenza]
Dato $G = \grafo$ la relazione di congiungibilità in $G$ su $V$ è una relazione di equivalenza su $V$:
\begin{enumerate}
\item$\textit{(riflessività) }\qquad u\ \til\ u \hfill \forall u \in V$
\item$\textit{(simmetria) }\qquad (u\ \til\ v) \implica (v\ \til\ u) \hfill \forall v, w \in V$
\item$\textit{(transitività)}\qquad (u\ \til\ v) \land (v\ \til\ w) \implica (u\ \til \ w) \hfill \forall v, u, w \in V$
\end{enumerate}
Indicheremo la relazione d'equivalenza con $\til$.
\end{theorem}

\begin{proof}
Siano $u, v, w \in V$, $\til$ la relazione d'equivalenza. Vale:
\begin{enumerate}
\item è vera in quanto $(u)$ è un cammino che congiunge $u$ a $u$.
\item è vera in quanto se $u\ \til\ v$ esiste una passeggiata $P = (v_0,\ldots, v_n)$ tale che $u = v_0$ e $v = v_n$. Ma allora $P^{\pr} = (v_n, v_{n-1},\ldots ,v_0)$ è una passeggiata, dove vertici consecutivi in $P$ lo sono anche in $P^{\pr}$ (anche se in ordine inverso).
\item è vera in quanto se $u\ \til\ v$ e $v\ \til\ w$ allora esistono due passeggiate $P_1 = (v_0,\ldots ,v_n), P_2 = (w_0,\ldots, w_m)$ dove $u = v_0,\ v = v_n = w_0,\ w = w_m$. Possiamo definire una terza passeggiata $P_3 = (v_0,\ldots ,v_n,w_1,\ldots ,w_m)$ costruita come unione delle precedenti; $P_3$ è una passeggiata in quanto vertici consecutivi in $P_3$ lo sono o in $P_1$ o in $P_2$, e i primi e ultimi vertici della passeggiata sono rispettivamente $u$ e $w$.
\end{enumerate}
\end{proof}

\section{Relazione fondamentale nei grafi finiti e lemma delle strette di mano}
\begin{theorem}[Relazione fondamentale tra $|\Eps(G)|$ e $deg(Gi)$ in un grafo finito]
Sia $G = \grafo$ un grafo finito. Vale:
$$ 2 \cdot | \mathlarger{\mathlarger{\epsilon}} | \ = \ \sum_{v\in V} deg_G (v) $$
\end{theorem}

\begin{proof}
Siano $v_1, v_2,\ldots ,v_n$ i vertici di G, $e_1, e_2,\ldots, e_k$ i lati di G (dove $k \coloneqq | \Eps |$). Sia
\[
M_{ij} \coloneqq \left\{
\setlength\arraycolsep{2pt}
\begin{array}{ccc} 0 &\qquad v_i \notin \ \Eps_j &\qquad \forall i \in \{1, 2, \ldots , n\} \\ \relax
1 &\qquad v_i \in \ \Eps_j &\qquad \forall j \in \{1, 2, \ldots ,k \} 
\end{array}\right
.
\]
dove $i$ rappresenta l'indice sul numero dei vertici e $j$ l'indice sul numero dei lati. Vale, grazie alla proprietà commutativa delle somme:
$$ (1)\ \ \sum_{i=1}^n \sum_{j=1}^k m_{ij} \quad = \quad \sum_{j=1}^k\sum_{i=1}^n m_{ij}\ \ (2)$$
dove $(1)$ rappresenta una somma di sommatorie con un vertice $i$ fissato; in ciascuna somma, si somma $1$ se un lato contiene il vertice fissato, $0$ se ciò non accade. Ma ciò non è altro che il grado del dato vertice; $(2)$ invece somma $k$ volte una sommatoria con un lato $j$ fissato, dove viene sommato $1$ tante volte quante un vertice appartiene a un dato lato, ovvero 2. Infine vale:
\begin{align*}
 \sum_{v\in V} deg(v) \ =& \ 2k \\
 =&\ 2 | \mathlarger{\mathlarger{\epsilon}} |
 \end{align*}
\end{proof}

\begin{theorem}[Lemma delle strette di mano]
In un grafo $G = \grafo$ finito il numero di vertici di grado dispari è pari.
\end{theorem}

\begin{proof}
Sia $G = \grafo$. Vale, grazie alla relazione fondamentale tra lati e gradi di un grafo:
\begin{align*}
2 | \mathlarger{\mathlarger{\epsilon}} | \ &=\ \sum_{v\in V} deg (v) \\
2 | \mathlarger{\mathlarger{\epsilon}} | \ &= \ \underbrace{\sum_{v\in V} deg (v)}_{\text{deg(v) pari}} + \ \underbrace{\sum_{v\in V} deg (v)}_{\text{deg(v) dispari}} \\
2 | \mathlarger{\mathlarger{\epsilon}}| - \ \underbrace{\sum_{v\in V} deg (v)}_{\text{deg(v) pari}} \ &= \ \underbrace{\sum_{v\in V} deg (v)}_{\text{deg(v) dispari}} 
\end{align*}
Allora la somma dei vertici con grado dispari deve essere pari perché differenza di un numero pari e una somma di numeri pari.
\end{proof}

\section{Teorema di caratterizzazione degli alberi finiti}
\begin{theorem}[Teorema di caratterizzazione degli alberi finiti mediante la formula di Eulero]
Sia $T = \grafo$ un grafo finito. Le seguenti affermazioni sono equivalenti:
\begin{enumerate}
\item $T$ è un albero
\item $\forall v, v^{\pr} \in V, \exists! \text{ cammino da } v \text{ in } v^{\pr}$
\item $T$ è connesso e $\forall e \in \Eps,\ T-e \coloneqq (V,\ \Eps \setminus \{e\})$ è sconnesso
\item $T$ non ha cicli e $\forall e \in \binom{V}{2} \setminus \Eps, \ T + e \coloneqq (V, \ \Eps \ \cup \ \{e\})$ ha almeno un ciclo
\item $T$ è connesso e $|V| - 1 = |\Eps|$.
\end{enumerate}
\end{theorem}

\renewcommand\qedsymbol{$\square$}
\begin{proof}\ \\
$(1 \implica 5).$ Procediamo per induzione su $|V(T)|$.
\baseinduzalbero{1}
\\Vale $|\Eps(T)| = 0 = |V(T)| - 1$.
\induzalbero{2}\nopagebreak
\\Sia $T$ un qualsiasi albero con $|V(T)| \ge 2$. Dimostriamo che vale la proprietà $(5)$. Essendo $T$ un albero, $\exists$ almeno una foglia $v \in T$. Consideriamo ora $T - v$: è ancora un albero, dove
\begin{align*}
|V(T-v)| &= |V(T)| - 1 \\
| \mathlarger{\mathlarger{\epsilon}} (T-v) | &= |\mathlarger{\mathlarger{\epsilon}} (T) | -1 
\end{align*}
Vale, per ipotesi induttiva:
\begin{align*}
|V(T-v) | -1 &= |\mathlarger{\mathlarger{\epsilon}}(T-v)|,
\\ \qquad\qquad |V(T)| -1 -1 &= |\mathlarger{\mathlarger{\epsilon}}(T) | -1
\end{align*}
\end{proof}

\renewcommand\qedsymbol{$\blacksquare$}
\begin{proof}[$(1 \Longleftarrow 5)$]
Procediamo per induzione su $|V(T)|$.
\baseinduzalbero{1}
\\ Un grafo con 1 vertice e 0 lati è un albero per definizione.
\induzalbero{2}\nopagebreak
\\ Sia $T$ un grafo connesso che soddisfa la formula di Eulero. Supponiamo per assurdo che $T$ non abbia foglie, ovvero che $deg(v) \ge 2 \ \ \forall v \in V(T)$. Allora
\begin{align*}
|V(T)| - 1 &= \frac{1}{2} \sum_{v \in V} deg(V) \\ 
2\ |V(T) | -2 &= \sum_{v \in V} deg(V) \ge \underbrace{2 \ |V(T)|}_{deg(V) \ge 2 \ \forall v}
\end{align*}
che è un assurdo. Allora $T$ ha almeno una foglia. Se consideriamo $v \in V(T)$ foglia, $T-v$ è ancora connesso e vale Eulero. Allora per ipotesi induttiva $T - v$ è un albero $\implica$ $T$ è un albero.
\end{proof}

\section{Teorema di esistenza degli alberi di copertura}
\begin{theorem}[Teorema di esistenza degli alberi di copertura per un grafo finito]
Ogni grafo connesso ammette almeno un albero di copertura.
\end{theorem}

\begin{proof}
Determiniamo
$$ \mathcal{T} \coloneqq \{ T \ | \ \text{$T$ è un sottografo di G, $T$ è un albero} \}$$
Sia $\overline{T} \in \mathcal{T} : |V(\overline{T}) | \ge |V(T)| \qquad \forall T \in \overline{T}$.
\\Osservo che $\overline{T} \ne \emptyset$ in quanto se $v \in V(G)$ allora $(v, \emptyset) \in \mathcal{T}$. Proviamo che $V(\overline{T}) = V(G)$ ovvero che $\overline{T}$ è un albero di copertura.\\
Usando la connessione di G, è possibile determinare un vertice $w \in V(G) \setminus V(\overline{T})$ e un vertice $u \in V(\overline{T})$ tali che $\{u, w\} \in\  \Eps(G)$. Ma allora possiamo definire
$$ \overline{\overline{T}} \in \mathcal{T},\ \overline{\overline{T}} \coloneqq(V(\overline{T}) \cup \{w\}, \mathlarger{\mathlarger{\epsilon}}(\overline{T}) \cup \{u, w\})$$
che è chiaramente un albero, ma che va in contraddizione con la massimalità dei vertici di $\overline{T}$.
\end{proof} 
\end{document}



